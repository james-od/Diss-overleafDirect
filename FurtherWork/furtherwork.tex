%%%%%%%%%%%%%%%%%%%%%%%%%%%%%%%%%%%%%%%%%%%%%%%%%%%%%%%%%%%%%%%%%%%%%%%%%%%%%%%%
%2345678901234567890123456789012345678901234567890123456789012345678901234567890
%        1         2         3         4         5         6         7         8
% THESIS CHAPTER

\chapter{Usage Scenario}

\section{Scenario 1} 

\section{Scenario 2} 

\section{Scenario 3} 


\chapter{Conclusion}

\section{Potential further work} 
Creating visualisations is a highly iterative and experimental process, so naturally many ideas have been produced for improvements that could be made. 
One of the most immediately useful would be implementing automatic point of interest detection. Specific frames of the network where most measures changed noticably compared with previous values would be highlighted on the timeline.
\newline

Another improvement mentioned during the feedback meeting (and briefly considered before) was enabling graphs to be exported as a png to allow for easy insertion into academic papers. This would encourage adoption and improve quality of life for end users.
\newline

Adding the ability to move the databar into a separate window, which could then be dragged to a separate screen - creating more space for the node-link diagram and allowing more graphs in the databar to be visualised at once. 
\newline

Running a think-aloud study or similar user trial could be useful to discover more about how effective the current implementation is and provide insight into further improvements.
\newline

Adding more measures is an obvious improvement - particularly local dynamic measures as they aren't found in other tools. An example could be a measure to add a sense of node 'promiscuity' - if volatility focuses on how the lifespans of node-pairs vary over time then 'promiscuity' would focus more on how often those edges tend to be with the same nodes. A node that had a few fairly rigid connections but also made a new connection every time step would have relatively low volatility but high promiscuity. A node that erratically gained and lost connections but only to a select few nodes would have high volatility but low promiscuity.

Could add selecting/locking a time period instead of using the entire period for the greyed out backing circles for the nodes

\section{Evaluation of work completed} 
The system works well in the described usage scenarios. The complementary approach taken means that...
